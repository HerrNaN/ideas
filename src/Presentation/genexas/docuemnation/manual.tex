\documentclass{article}
\begin{document}
\title{Documentation to the Online Exercise Assistant}
\author{Sylvia Stuurman}
\date{july 2008}
\maketitle
At this moment, there are three different ways to use the on-line Exercise Assistant.

\section{The HTML}
The HTML of the Exercise Assistant shows several elements which can be questioned and modified by Javascript, as we will see below.

The elements are:
\begin{itemize}
\item The \textbf{New Exercise} button, which will invoke a \textbf{generate} function,
\item The \textbf{Rules}-, \textbf{Help}- and \textbf{About}-button, which will cause three different HTML fragment to show,
\item The \textbf{Exercise Area} where the exercise will be shows (not editable by the user),
\item The \textbf{Work Area} where the user may rewrite the expression,
\item The \textbf{Progress Area} which will show how many steps will be needed minimally,
\item The \textbf{Hint}-button which will invoke the hint function,
\item The \textbf{Next}-button which will invoke the next function,
\item The \textbf{Derivation}-button which will invoke the derivation function,
\item The \textbf{Submit}-button which will invoke the feedback function,
\item The \textbf{Copy}-button (only visible when there is something to copy) which will invoke the copy function,
\item The \textbf{Back}-button (only visible when there is something to go back to) which will invoke the back function,
\item The \textbf{Forward}-button (only visible when there is something to go forward to) which will invoke the forward function,
\item The \textbf{Ready}-button which will invoke the ready function,
\item The \textbf{History Area} which will show the valid steps,
\item The \textbf{Feedback Area} whic will show the feedback. There are radio buttons with which the user nmay choose between seeing all feedback, or only the current feedback,
\item In the OU version, there are also page elements for the student number.
\end{itemize}

\section{How to invoke the Exercise Assistant}
\subsection{Generic}
\textbf{http://ideas.cs.uu.nl/genexas} \\
shows links to the possible types of exercises that are
offered by the Strategy-based feedbackservices. The links are hard-coded, but when an
extra service would send a list of types of exercises, this service could be used to 
generate the list of links automatically.

The links now look like:

\textbf{$<$ a href=``generic.php?exercisekind=Proposition to DNF''$>$Proposition Logic$<$/a$>$} and

\textbf{$<$a href=``generic.php?exercisekind=Relational Algebra''$>$Relational Algebra$<$/a$>$}.


The file ``generic.php'' inludes standard help-, about- and rules files, includes a file which sets the language into English, and includes the file ``common/generic.php''.

This file contains the HTML which produces a page with the Exercise Assistant. The variable exercisekind in the url is used to set the value of a Javascript variable called exercisekind, which will be used for the calls to the feedback services.

\subsubsection{Language}
The language now is default set to English, but using the language options of the Apache server, it would be easy to switch to multilanguage. In that case, instead of a single index.php, there would be an index.en.php and an index.nl.php (etc), and instead of including the file with English names, the appropriate language file would be included.

The file structure is:\\
/genexas/index.php \\
/genexas/generic.php \\
/genexas/common/en.php \\
/genexas/common/nl.php \\
/genexas/common/generic.php \\
/genexas/common/en/about.php \\
/genexas/common/en/help.php \\

Changing to multilanguage would result in\\
/genexas/index.en.php \\
/genexas/index.nl.php \\
/genexas/generic.php (to be called with a language variable)\\
/genexas/common/en.php \\
/genexas/common/nl.php \\
/genexas/common/generic.php \\
/genexas/common/en/about.php \\
/genexas/common/en/help.php \\
/genexas/common/nl/about.php \\
/genexas/common/nl/help.php \\

\subsection{Specific}
Another way to use the Exercise Assistant is through two specific URL's: 

\textbf{http://ideas.cs.uu.nl/genexas/logic/todnf/index.php} \\ (which makes use of the multilanguage facility; because only English is fully implemented, it is safest to use http://ideas.cs.uu.nl/genexas/logic/todnf/index.en) or 

\textbf{http://ideas.cs.uu.nl/genexas/math/relationalgebra/index.php} \\ (for which the same applies, so preferrably use http://ideas.cs.uu.nl/genexas/math/relationalgebra/index.en).

The difference with the so-called generic url is, that it is possible to integrate domain-specific HTML and Javascript, for instance to enhance the user interface by providing real mathematical symbols instead of ASCII characters for input and output.

\section{OU specific}
At last, there is the possibility to use the Exercise Assistant for OU students, asking them for their student number:

\textbf{http://ideas.cs.uu.nl/genexas/logic/todnf/ou/}

In this case, the student number will be sent with each call to the strategy-based feedback services.

\section{Javascript}
\subsection{/genexas/common/javascript/services.js}
This file contains functions to call the strategy-services. Each function requires the input which is needed to call the service, and a callback function. The services are called through an asynchronous Ajax call, and the callback function will be called when the result has arrived.

The functions are:
\begin{itemize}
\item function ss\_generate(number, callback) 
\item function ss\_getReady(state, callback) 
\item function ss\_getHint(location, state, callback) 
\item function ss\_getNext(state, callback)
\item function ss\_getDerivation(state, callback)
\item function ss\_getRemaining(state, callback) 
\item function ss\_getFeedback(state, newexpression, callback)
\end{itemize}

\subsection{/genexas/common/javascript/communication.js}
The functions in this file function as a mediator between the elements of the HTML page and the services.js-functions. 

For each action that the user interface allows to, there are a pair of functions: one that knows where to get the appropriate paramteres from, and one that knows how to display the results. The display function is used as the callback function when calling a function from services.js.

The pairs of functions are:

\begin{itemize}
\item function generate() and function displayExercise(state)
\item function getHint() and function displayHint(listOfRules)
\item function getNext() and  function displayNext(rule, location, state) 
\end{itemize}
\end{document}